\documentclass[12pt, a4paper]{article}

\usepackage{amssymb,amsmath,latexsym}
\usepackage[english]{babel}
\usepackage{hhline}

\newcommand{\code}[1]{{\tt #1}}

\newenvironment{nitemize}{
  \newdimen\oldparindent
  \oldparindent=\parindent
  \begin{itemize}
  \itemindent=-\oldparindent
}{
  \end{itemize}
}

%\newcommand{\codeblockspace}{\vspace{12pt}}

% begin/end a code block
\newcommand{\codeblockbegin}{\begin{flushleft}\begin{minipage}{\textwidth}}
\newcommand{\codeblockend}{\end{minipage}\end{flushleft}}


\begin{document}

\sloppy

\title{The Resources Format}
%\date{}
\author{Ingo Weinhold (bonefish@users.sf.net)}

\maketitle
\tableofcontents


\section{Introduction}
\label{introduction}

Resources provide a means to store structured but flat data in files. Unlike
attributes resources are part of the file contents and thus do not require a
special file system handling, but rather a special file format.
On the one hand there are formats of files that exclusively contain resources
(resource files), on the other hand these are file formats extended to
additionally contain resources -- namely the ELF and PEF object formats.
In either case the format of the chunk of data that frames the resources
themselves is the same. We call it the resources format.

Section \ref{file-formats} explains how the resources format is embedded in
different file formats. Section \ref{resources-format} discusses the resources
format itself. In section \ref{implementations} we focus on robustness of
resources reading/writing implementations.
The final section says some words about the status of the information provided
by this document.



\section{File Formats}
\label{file-formats}

In all file formats described in this section the resources are being located
at the end of the files. They are completely independent of their location.


\subsection{x86 Resource Files}

x86 resource files introduce the least overhead. The resources start directly
after the magic number identifying the file format:
%
\codeblockbegin
\begin{verbatim}
  const char kX86ResourceFileMagic[4] = { 'R', 'S', 0, 0 };
  const uint32 kX86ResourcesOffset    = 0x00000004;
\end{verbatim}
\codeblockend
%
The resources start at \code{kX86ResourcesOffset}.


\subsection{PPC Resource Files}

PPC resource files begin with a PEF container header, after which the
resources start.
%
\codeblockbegin
\begin{verbatim}
typedef char PefOSType[4];

struct PEFContainerHeader {
    PefOSType tag1;
    PefOSType tag2;
    PefOSType architecture;
    uint32    formatVersion;
    uint32    dateTimeStamp;
    uint32    oldDefVersion;
    uint32    oldImpVersion;
    uint32    currentVersion;
    uint16    sectionCount;
    uint16    instSectionCount;
    uint32    reservedA;
};
\end{verbatim}
\codeblockend
%
\codeblockbegin
\begin{verbatim}
  const char kPEFFileMagic1[4]        = { 'J', 'o', 'y', '!' };
  const char kPPCResourceFileMagic[4] = { 'r', 'e', 's', 'f' };
  const uint32 kPPCResourcesOffset    = 0x00000028;
\end{verbatim}
\codeblockend

\begin{nitemize}
\item{\code{tag1}:
  Must be \code{kPEFFileMagic1}.}
\item{\code{tag2}:
  Must be \code{kPPCResourceFileMagic}.}
\item{All other fields must be set to 0.}
\end{nitemize}

\noindent
The resources start at \code{kPPCResourcesOffset}.


\subsection{ELF Object Files}

In an ELF file, resources are appended to rather than contained in the
regular data of the file. That is adding resources to an existing ELF file
will not cause any modification to its data (i.e. ELF header, program header
table, section header table or sections), but will enlarge the file by some
alignment padding and, of course, the resources themselves.

Therefore two values have to be known: The size of the actual ELF file and the
block size to which the resources must be aligned. As ELF files do not contain
a size field, it has to be deduced, where the file ends. This end offset is
supposed to be the maximum of the end offsets of ELF header, program header
table (if any), section header table (if any), sections and segments.

The block size to which the resources have to be aligned is the maximum of
\code{kELFMinResourceAlignment} and the alignments of the segments in the file.
%
\begin{verbatim}
  const uint32 kELFMinResourceAlignment = 32;
\end{verbatim}
%
The data used for the padding between the end of the actual ELF data and the
beginning of the resources may be arbitrary.


\subsection{PEF Object Files}

Similar to ELF files the resources are simply appended to the regular data of
a PEF file, but they are not aligned to any value. That is the resources
start directly after the last PEF section without any padding.
As no field exists, that tells about the size of the PEF container (the
regular data), it has to be deduced by iterating through the PEF section
headers.



\section{The Resources Format}
\label{resources-format}

This section describes the resources format. After a subsection that outlines
their general layout, it follow subsections discussing the major parts.

A general remark regarding the byte ordering: Resources have no standard
endianess, that is the resources created by little endian and big endian
machines differ. Usually it should be possible, to deduce the used endianess
from the type of the file. x86 resource files contain little endian, PPC
resource files big endian data. The endianess of an ELF file is encoded in
its header.

As there is in fact no good reason to have different resource file formats,
even if they differ only in the format of the header (see section
\ref{file-formats}), it may be decided to use the x86 resource file format
also for big endian machines. Therefore the endianess may be deduced by the
first field of the resources header (\code{rh\_resources\_magic}, see
subsection \ref{resources-header}).


\subsection{Resources Layout}

The layout of the resources in a file is shown in figure
\ref{fig:resources-layout}.

\begin{figure}[h!tb]
  \begin{center}
  \begin{tabular}{|c|c|c|}
  \hline
                & \multicolumn{2}{c|}{resources header}\\
  \hhline{|~==|}
  admin section &               & index section header \\
  \hhline{~~|-|}
                & index section & resource index \\
  \hhline{~~|-|}
                &               & padding \\
  \hhline{:===:}
  \multicolumn{3}{|c|}{unknown section}\\
  \hhline{:===:}
  \multicolumn{3}{|c|}{data section}\\
  \hhline{|~~~|}
  \hhline{:===:}
  \multicolumn{3}{|c|}{info section}\\
  \hline
  \end{tabular}
  \end{center}
  \caption{The Resources Layout.}
  \label{fig:resources-layout}
\end{figure}

\noindent
There are four sections:
%
\begin{itemize}
\item{An administrative section which comprises the resources header and the
  resource index subsection. The latter locates all other data in the file.
}
\item{An unknown section, whose purpose is (unsurprisingly) unknown, but which
  seems to be unused, always containing the same data.
}
\item{A data section holding the actual resource data.
}
\item{An info section, which provides aditional information for each resource,
  such as type, id and name.
}
\end{itemize}


\subsection{Resources Header}
\label{resources-header}

The resources header has the following structure:
%
\codeblockbegin
\begin{verbatim}
  struct resources_header {
      uint32 rh_resources_magic;
      uint32 rh_resource_count;
      uint32 rh_index_section_offset;
      uint32 rh_admin_section_size;
      uint32 rh_pad[13];
  };
\end{verbatim}
\codeblockend
%
\codeblockbegin
\begin{verbatim}
  const uint32 kResourcesHeaderMagic          = 0x444f1000;
  const uint32 kResourceIndexSectionOffset    = 0x00000044;
  const uint32 kResourceIndexSectionAlignment = 0x00000600;
\end{verbatim}
\codeblockend

\begin{nitemize}
\item{\code{rh\_resources\_magic}:
  Must be \code{kResourcesHeaderMagic}.
}
\item{\code{rh\_resource\_count}:
  Specifies the number of resources stored in this file. May be 0.
}
\item{\code{rh\_index\_section\_offset}:
  Specifies the offset of the resource index section relative to the beginning
  of the resources. An alternative interpretation may be the size of the
  resources header.
  Must be \code{kResourceIndexSectionOffset}.
}
\item{\code{rh\_admin\_section\_size}:
  Specifies the size of the administrative section.
  Must be \code{kResourceIndexSectionOffset} plus a multiple of
  \code{kResourceIndexSectionAlignment}.
}
\item{\code{rh\_pad}:
  Padding. \code{0x00000000} words.
}
\end{nitemize}


\subsection{Resource Index Section}
\label{resources-index}

The resource index section starts with a header, it follows a table of
\code{resource\_index\_entry} structures, that locates the data of each
resource, and the section ends with a special padding.

\noindent
The resource index header has the following structure:
%
\codeblockbegin
\begin{verbatim}
  struct resource_index_section_header {
      uint32 rish_index_section_offset;
      uint32 rish_index_section_size;
      uint32 rish_unused_data1;
      uint32 rish_unknown_section_offset;
      uint32 rish_unknown_section_size;
      uint32 rish_unused_data2[25];
      uint32 rish_info_table_offset;
      uint32 rish_info_table_size;
      uint32 rish_unused_data3;
  };
\end{verbatim}
\codeblockend

\begin{verbatim}
  const uint32 kUnknownResourceSectionSize = 0x00000168;
\end{verbatim}
%
\begin{nitemize}
\item{\code{rish\_index\_section\_offset}:
  Specifies the offset of the resource index section relative to the beginning
  of the resources. An alternative interpretation may be the size of the
  resources header.
  Must be \code{kResourceIndexSectionOffset}.
}
\item{\code{rish\_index\_section\_size}:
  Specifies the size of the resource index section.
  Must be a multiple of \code{kResourceIndexSectionAlignment}.
}
\item{\code{rish\_unused\_data1}:
  Contains special data as described in section \ref{resources-unknown}.
}
\item{\code{rish\_unknown\_section\_offset}:
  Specifies the offset of the unknown section relative to the beginning
  of the resources.
  Must be the same value as given in the resources header for
  \code{rh\_admin\_section\_size}.
}
\item{\code{rish\_unknown\_section\_size}:
  Specifies the offset of the unknown section relative to the beginning
  of the resources.
  Must be \code{kUnknownResourceSectionSize};
}
\item{\code{rish\_unused\_data2}:
  Contains special data as described in section \ref{resources-unknown}.
}
\item{\code{rish\_info\_table\_offset}:
  Specifies the offset of the resource info table relative to the beginning
  of the resources.
}
\item{\code{rish\_info\_table\_size}:
  Specifies the size of the resource info table.
}
\item{\code{rish\_unused\_data3}:
  Contains special data as described in section \ref{resources-unknown}.
}
\end{nitemize}

Directly, without padding, it follows a table of \code{resource\_index\_entry}
structures. The number of entries in the table is the number of resources
stored in the file, that is the value specified by the
\code{rh\_resource\_count} member of the resources header. Since the entries
are stored without padding, the size of the table is exactly the product of
the size of \code{resource\_index\_entry} and the number of resources.
If the latter is 0, the table takes no space.
%
\codeblockbegin
\begin{verbatim}
  struct resource_index_entry {
      uint32 rie_offset;
      uint32 rie_size;
      uint32 rie_pad;
  };
\end{verbatim}
\codeblockend
%
\begin{nitemize}
\item{\code{rie\_offset}:
  Specifies the offset of the resource data relative to the beginning
  of the resources.
}
\item{\code{rie\_size}:
  Specifies the size of the resource data.
}
\item{\code{rie\_pad}:
	Padding. Must be \code{0x00000000}.
}
\end{nitemize}

Since the size of the resource index section must be a multiple of
\code{kResourceIndexSectionAlignment}, some padding may be needed at the end
of this section. How this padding looks like is described in section
\ref{resources-unknown}.


\subsection{Unknown Section}
\label{resources-unknown}

The meaning of this section is unknown. It does not seem to be used at all.
It always contains the same data given by \code{kUnusedResourceDataPattern}:
%
\codeblockbegin
\begin{verbatim}
  const uint32 kUnusedResourceDataPattern[3] = {
      0xffffffff, 0x000003e9, 0x00000000
  };
\end{verbatim}
\codeblockend
%
In section \ref{resources-index} some members where named \code{unused\_data}.
These fields contain the same kind of data. To understand what the value for a
certain field of this type is, it may help to imagine, that before the
resources are written to a file, the space they will take is filled with the
pattern specified by \code{kUnusedResourceDataPattern}, and that only those
fields are written that are not unused. Thus the original pattern can be seen
through at the unused locations.

To be precise: Let \verb|uint32 resources[]| be the resources and
\code{index} the index of an unused field in \code{resources}, then it holds:
%
\begin{verbatim}
  resources[index] == kUnusedResourceDataPattern[index % 3];
\end{verbatim}
%


\subsection{Resource Info Table}
\label{resources-infotable}

The resource info table features exactly one entry for each resource.
Such an entry (resource info) specifies the ID and name of a
resource. Subsequent infos for resources of the same type are collected in
a block that starts with a type field.

The following grammar specifies the layout of the resource info table.
Nonterminals start with an upper case, terminals with a lower case letter.
%
\begin{verbatim}
  ResourceInfoTable      ::= [ ResourceBlockList ]
                             ResourceInfoSeparator
                             ResourceInfoTableEnd

  ResourceBlockList      ::= ResourceBlock
                             [ ResourceInfoSeparator
                               ResourceBlockList ]

  ResourceBlock          ::= type ResourceInfoList

  ResourceInfoList       ::= ResourceInfo [ ResourceInfoList ]

  ResourceInfo           ::= id index name_size name

  ResourceInfoSeparator  ::= 0xffffffff 0xffffffff

  ResourceInfoTableEnd   ::= check_sum 0x00000000
\end{verbatim}
%
The relevant structures follow:
%
\codeblockbegin
\begin{verbatim}
  struct resource_info_block {
      type_code     rib_type;
      resource_info rib_info[1];
  };
\end{verbatim}
\codeblockend
%
\begin{nitemize}
\item{\code{rib\_type}:
  Specifies the type of the resources in the block.
}
\item{\code{rib\_info}:
  Is the first resource info of the block. More infos may follow.
}
\end{nitemize}
%
\codeblockbegin
\begin{verbatim}
  struct resource_info {
      int32  ri_id;
      int32  ri_index;
      uint16 ri_name_size;
      char   ri_name[1];
  };
\end{verbatim}
\codeblockend

\begin{verbatim}
  const uint32 kMinResourceInfoSize = 10;
\end{verbatim}
%
\begin{nitemize}
\item{\code{ri\_id}:
  Specifies the ID of the resource.
}
\item{\code{ri\_index}:
  Specifies the index of the resource this resource info refers to.
}
\item{\code{ri\_name\_size}:
  Specifies the size of the resource name. May be 0 -- then the resource does
  not have a name and \code{ri\_name} has a size of 0.
}
\item{\code{ri\_name}:
  Specifies the name of the resource. The name must be null terminated.
  \code{ri\_name\_size} specifies the size of this field (including the
  terminating null). If it is 0, the resource does not have a name and
  \code{ri\_name} is empty, i.e. has size 0.
}
\item{\code{kMinResourceInfoSize}:
  Is the minimal size of a resource info. That is the size it has, if the
  resource does not have a name.
}
\end{nitemize}
%
\codeblockbegin
\begin{verbatim}
  struct resource_info_separator {
      uint32 ris_value1;
      uint32 ris_value2;
  };
\end{verbatim}
\codeblockend
%
\begin{nitemize}
\item{\code{ris\_value1}:
  Specifies the first word of the separator.
  Must be \code{0xffffffff}.
}
\item{\code{ris\_value2}:
  Specifies the second word of the separator.
  Must be \code{0xffffffff}.
}
\end{nitemize}
%
\codeblockbegin
\begin{verbatim}
  struct resource_info_table_end {
      uint32 rite_check_sum;
      uint32 rite_terminator;
  };
\end{verbatim}
\codeblockend
%
\begin{nitemize}
\item{\code{rite\_check\_sum}:
  Contains the check sum for the resource info table. The check sum is
  calculated from all bytes of the resource info table not including
  \code{rite\_check\_sum} and \code{rite\_terminator}. The data are grouped
  into four byte blocks, which are interpreted as big endian unsigned words
  and summed up, ignoring carry. If the number of bytes to be considered is
  not dividable by four, the remaining bytes are interpreted as the lower
  bytes of a big endian unsigned word (the upper byte(s) set to 0).
}
\item{\code{rite\_terminator}:
  Terminates the resource info table.
  Must be \code{0x00000000}.
}
\end{nitemize}



\section{Implementations}
\label{implementations}

Code that writes resources should strictly stick to the specification
presented in the preceding sections to achieve maximal compatibility.

Resources reading implementations may tolerate certain deviations that
for instance happen to occur in several files of the BeOS R5 distribution
and that are handled gracefully by xres and QuickRes. It follows a, possibly
incomplete, list:
%
\begin{itemize}
\item{The third and fourth byte of the x86 resource file magic (the 0 bytes)
  may be arbitrary bytes.}
\item{\code{rh\_resource\_count} may be unreliable. The resource index table
  should be read until its end, which is either marked by the unused data
  pattern (see section \ref{resources-unknown}) or at the latest by the
  beginning of the unknown section.}
\item{The resource info table may contain entries for indices that are out
  of range, i.e. greater than the number of resources induced by the resource
  index table. Those entries should be ignored.}
\item{The resource info table may contain multiple entries for an index.
  Any such entry after the first one should be ignored.}
\item{The resource info table may not contain an entry for an index.
  The respective resource should be ignored.}
\item{The resource info table may not contain a \code{ResourceInfoTableEnd}
  (see section \ref{resources-infotable}) and thus no check sum. The table
  should be accepted nevertheless. Note, that a table containing a wrong check
  sum is {\em not} to be accepted.}
\end{itemize}



\section{Status of this Document}
\label{status}

The information contained in this document are obtained by analyzing
resources-containing files created or modified by tools available for BeOS R5,
namely QuickRes and xres. They are incomplete and may even be partially wrong,
where being based on incorrect assumptions.

\noindent
It follows a list of items with a low degree of reliance:
\begin{itemize}
\item{Resources alignment in ELF files: Several tests with linker object files
  have shown, that QuickRes aligns their resources offset to 32 bytes.
  For executables on the other hand the alignment was always 4096, which is
  the usual memory page size of current x86 architectures and therefore the
  preferred program segment alignment. From these two information it has been
  deduced, that the alignment is, if present, the maximum of
  the segment alignments to be found in the program header table,
  but at minimum 32.}
\item{The resources layout: The general layout of the resources is not very
  well understood. The layout presented in figure \ref{fig:resources-layout}
  resulted from the attempt to assign all the fields a reasonable meaning, but
  in fact not even the exact length and meaning of the fields of the resources
  header is unclear. The same holds for the resource index section header.}
\item{The unknown section: The contents of the unknown section and of unknown
  fields is base on educated guesses.}
\end{itemize}

\end{document}
