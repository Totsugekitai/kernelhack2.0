\documentclass[main]{subfiles}

\begin{document}

\section{実験課題10:その他 - eBPFを用いたトレース}

この課題では、カーネルの機能であるeBPFを用いたカーネル内の処理のトレースを行う。

\subsection{eBPFの概要}

まずはeBPFの概要を記す。

\subsubsection{eBPFとは}

eBPFは extended Berkeley Packet Filter の頭文字を取ったもので、パケットフィルタとして開発されたBPFを拡張して作られた機能である。
カーネル内の様々な関数・システムコール・処理をトレースすることができる。
またカーネル内で動作する仮想マシンの中でBPFアプリケーションは動くため、比較的安全性が高い。

\subsection{トレーサーの実装と実行結果}

今回はシステムコールのトレーサーを作成した。

\subsubsection{トレーサーの実装}

Listing\ref{tracer}にプログラムを示す。

\lstinputlisting[label=tracer,caption=トレーサーのプログラム]{../userland/modify_bpf.bt}

このトレーサーはプロセスID、プロセス名、システムコール番号を1行に出力する。

\subsubsection{実行結果}

Listing\ref{tracer_exec}のようにトレーサーを実行していくつかコマンドを実行してみる。

\begin{lstlisting}[label=tracer_exec,caption=トレーサーを実行していくつかコマンドを打ってみる]
[root@archlinux host]# bpftrace modify_bpf.bt > bpf.log &
[1] 171
[root@archlinux host]# ls
a.out  bpf.bt   bpftrace  logfuse.c      module.c     syscall.c  use_getppid
bcc    bpf.log  logfuse   modify_bpf.bt  my_module.h  test       use_getppid.c
[root@archlinux host]# cd ..
[root@archlinux ~]# mkdir empty_dir
[root@archlinux ~]# kill 171
\end{lstlisting}

トレースログが\texttt{bpf.log}に出力されるので表示すると、Listing\ref{tracer_log}のようになる。

\lstinputlisting[label=tracer_log,caption=ログ]{../userland/bpf.log}

確かにログが取れていることが分かった。

\end{document}
