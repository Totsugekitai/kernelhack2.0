\documentclass[main]{subfiles}

\begin{document}

\section{実験課題3:システムコールの追加}

本課題では、次のような条件を満たすシステムコールを追加する。

\begin{itemize}
 \item ユーザ空間からは引数を2つ以上とる。
 \item 引数のうち1つは必ず番地を指定する。番地の先には可変長のデータがあるものとする。
 \item 結果を返す。結果はレジスタか、または与えられた番地に返す。
 \item 無効な引数とアクセス権に関してエラー処理を含む。
\end{itemize}

\subsection{Linuxのシステムコールの仕組み}

Linuxのシステムコールの仕組みについて説明する。

\texttt{kernel 5.7.10}のシステムコールは\texttt{arch/x86/entry/syscalls/syscall\_64.tbl}で登録されている。
Listing\ref{syscall_64tbl}に抜粋する。

\begin{lstlisting}[label=syscall_64tbl,caption=syscall\_64.tblの一部]
#
# 64-bit system call numbers and entry vectors
#
# The format is:
# <number> <abi> <name> <entry point>
#
# The __x64_sys_*() stubs are created on-the-fly for sys_*() system calls
#
# The abi is "common", "64" or "x32" for this file.
#
0	common	read			sys_read
1	common	write			sys_write
2	common	open			sys_open
3	common	close			sys_close
4	common	stat			sys_newstat
.
.
.
545	x32	execveat		compat_sys_execveat
546	x32	preadv2			compat_sys_preadv64v2
547	x32	pwritev2		compat_sys_pwritev64v2
\end{lstlisting}

Listing\ref{syscall_64tbl}では、一行ごとに、システムコール番号、ABI、名前、エントリポイントが記載されている。
ABIは\texttt{common, 64, x32}のどれかを指定する。
また、名前にはコンパイル時に\texttt{\_\_x64\_sys\_}の接頭辞が付けられる。

\end{document}
