\documentclass[main]{subfiles}

\begin{document}

\section{実験の環境}

\subsection{ホスト環境}

\begin{table}[]
\begin{tabular}{|r|l|}
\hline
計算機機種  & Lenovo ThinkPad L380                     \\ \hline
CPU    & Intel(R) Core(TM) i3-7020U CPU @ 2.30GHz \\ \hline
メモリ    & 32GB                                     \\ \hline
OS     & Arch Linux x86\_64                       \\ \hline
Kernel & 5.7.10-arch1-1                           \\ \hline
\end{tabular}
\end{table}

\subsection{ゲスト環境}

\begin{table}[]
\begin{tabular}{|r|l|}
\hline
VM環境   & QEMU               \\ \hline
割当メモリ  & 2GB                \\ \hline
OS     & Arch Linux x86\_64 \\ \hline
Kernel & 5.7.10             \\ \hline
\end{tabular}
\end{table}

\subsection{VMのセットアップ}

\subsubsection{VMの起動方法}
\label{how_to_vm_boot}

\texttt{QEMU}にListing\ref{qemu_option}のオプションをつけて起動する。

\begin{lstlisting}[label=qemu_option,caption=QEMUのオプション,language=make]
$ qemu-system-x86_64 -enable-kvm -s -kernel $(TOOLS)/bzImage -boot c -m 2049M -hda $(TOOLS)/image.img -append "root=/dev/sda rw console=ttyS0,115200 acpi=off nokaslr" -serial stdio -display none -virtfs local,path=$(SHARE),security_model=none,mount_tag=share --no-reboot
\end{lstlisting}

主要なオプションを解説する。

\texttt{-kernel}でカーネルイメージを指定する。

\texttt{-hda}で取り付けたいブロックデバイスを指定する。

\texttt{-append}でカーネルのブートオプションを指定する。
\texttt{root=/dev/sda}でrootfsを指定する。
\texttt{nokaslr}で\textit{Kernel Address Space Layout Randomization}を無効化する。

\texttt{-serial stdio}でゲストのシリアル入出力をホスト側の標準入出力に繋げ、\texttt{-display none}でGUIを表示しないようにする。

\texttt{-virtfs}で9Pプロトコルを用いたファイル共有の設定をする。
\texttt{path=(共有ディレクトリ)}で共有ディレクトリを指定する。

\subsubsection{rootfsの作成}

\ref{how_to_vm_boot}項で、QEMUのオプションの\texttt{-hda}で取り付けたブロックデバイスは、Linuxのrootfsと呼ばれる起動時にマウントされるファイルシステムである。
rootfsの作成はListing\ref{make_rootfs}の手順で行った。

\begin{lstlisting}[label=make_rootfs,caption=rootfsの作成]
$ qemu-img create image.img 8g
$ mkfs.ext4 image.img
$ mkdir mountpoint
$ sudo mount -o loop image.img mountpoint
$ sudo pacstrap mountpoint base linux linux-firmware
$ sudo umount mountpoint
\end{lstlisting}

まず\texttt{qemu-img create}コマンドでディスクイメージファイルを作成する。
ここでは8GBのディスクイメージを作成した。

次にディスクイメージを\texttt{ext4}ファイルシステムでフォーマットする。

さらに、\texttt{mountpoint}ディレクトリにディスクイメージをマウントし、\texttt{pacstrap}コマンドでrootfsに必要なパッケージをインストールし、rootfsを作成する。

最後にディスクイメージをアンマウントする。

\subsubsection{ホストとゲスト間でのファイル共有}

実験を円滑に進めるために、9Pプロトコルを用いてファイル共有を行う。

ゲスト側からホスト側のファイルにアクセスするためには、ゲスト側でListing\ref{file_sharing}のコマンドを実行すれば良い。

\begin{lstlisting}[label=file_sharing,caption=アクセス手順]
$ mkdir host
$ sudo mount -t 9p -o trans=virtio share ./host -oversion=9p2000.L
\end{lstlisting}

これにより、ゲスト側からホスト側のファイルにアクセスすることができる。

\end{document}
