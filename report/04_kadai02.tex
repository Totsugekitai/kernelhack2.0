\documentclass[main]{subfiles}

\begin{document}

\section{実験課題2:カーネルのリモートデバッグ}

本課題では、以降の実験を円滑に進めるために、カーネルをリモートデバッグできるようにする。

\subsection{GDBの使用方法}

GDBは、Listing\ref{gdb_comm}のように用いる。

\begin{lstlisting}[label=gdb_comm,caption=GDBの使用の様子,language=sh]
$ cd /path/to/linux
$ gdb ./vmlinux

(gdb) target remote localhost:1234
(gdb) b __x64_sys_getppid
(gdb) c

Breakpoint 2, get_current () at ./arch/x86/include/asm/current.h:15
15              return this_cpu_read_stable(current_task);
\end{lstlisting}

カーネルを\texttt{make}した際の生成物である\texttt{vmlinux}を引数に指定してGDBを起動する。

GDBを起動したら、\texttt{localhost}のポート\texttt{1234}番に対して接続を行う。
ここでVMの実行が停止する。

\texttt{break}コマンド(短縮コマンドは\texttt{b})でシステムコールの実体にbreakpointを仕掛ける。
ここでは\texttt{\_\_x64\_sys\_getppid}に対してbreakpointを仕掛けた。

\texttt{continue}コマンド(短縮コマンドは\texttt{c})でVMの実行を継続させる。
その後breakpointまで到達すると、VMの実行が停止する。
ここからGDBを用いてデバッグをする。

\subsection{GDBを用いたデバッグ}

Listing\ref{use_getppid}のC言語で書かれたプログラムを用いて、リモートデバッグが可能かどうか検証する。

\lstinputlisting[label=use_getppid,caption=GDBデバッグ検証に用いるプログラム,language=c]{../userland/use_getppid.c}

このプログラムは、ライブラリ関数\texttt{getppid}を呼び出して標準出力に出力するプログラムである。
このプログラムをゲスト環境で実行し、GDBを用いたデバッグが可能かどうか検証する。

Listing\ref{gdb_getppid}は、Listing\ref{use_getppid}のプログラムをゲスト上でコンパイルし、実行した際のGDBの様子である。

\begin{lstlisting}[label=gdb_getppid,caption=GDBの動作の様子,language=sh]
(gdb) target remote :1234
Remote debugging using :1234
default_idle () at arch/x86/kernel/process.c:676
676             trace_cpu_idle_rcuidle(PWR_EVENT_EXIT, smp_processor_id());
(gdb) b __x64_sys_getppid
Breakpoint 1 at 0xffffffff8107c020: file ./arch/x86/include/asm/current.h, line 15.
(gdb) c
Continuing.

Breakpoint 1, get_current () at ./arch/x86/include/asm/current.h:15
15              return this_cpu_read_stable(current_task);
(gdb)
\end{lstlisting}

うまくシステムコールをbreakできていることが確認できた。

\end{document}
